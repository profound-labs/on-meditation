
When developing \textit{Sam\=adhi},\footnote{\textit{Sam\=adhi} is one of the ways of practising meditation. The practice of \textit{Sam\=adhi} is focused on tranquillity and one-pointedness of mind, and entails the development of strong concentration. It leads to the various levels of \textit{jh\=ana}, meditative states of deep inner peace that go beyond the planes of ordinary human experience. \textit{Jh\=ana} is often translated as `absorption'.} fix your attention on the breath and imagine that you are sitting absolutely alone with no one and nothing else around to bother you. Develop this perception in the mind and sustain it until the mind completely lets go of the world outside. All that is left is simply the knowing of the breath entering and leaving. The mind must set aside the external world. Don't allow yourself to start thinking about this person who is sitting over here, or that person who is sitting over there. Don't give space to any thoughts that will give rise to confusion or agitation in the mind -- it is better to throw them out and be done with them. There is no one else here, you are sitting all alone. Develop this perception until all the other memories, perceptions and thoughts concerning other people and things subside, and you're no longer doubting or wondering about the other people or things around you. Then you can fix your attention solely on the in-breath and the out-breath. Breathe normally. Allow the in-breath and the out-breath to continue naturally, without forcing them to be longer or shorter, stronger or weaker than normal. Allow the breath to continue in a state of normality and balance, and then sit and observe it entering and leaving the~body.

Once the mind has let go of external mind-objects, this means you will no longer feel disturbed by the sound of traffic or other noises. You won't feel irritated with anything outside. Whether it's forms, sounds or whatever, they won't be a source of disturbance, because the mind won't be paying attention to them. The mind will become centred upon the breath.

If the mind is agitated by different things and you can't concentrate, try taking an extra deep breath until the lungs are completely full, and then release all the air until there is none left inside. Do this several times, then re-establish awareness and continue to develop concentration. When mindfulness is re-established, it is normal that for a period the mind will be calm, but then will change and become agitated again. When this happens, make the mind firm, take another deep breath and then expel all the air from your lungs. Fill the lungs to capacity again for a moment and then re-establish mindfulness on the breathing. Fix mindfulness on the \mbox{in-breath} and the out-breath, and continue to maintain awareness in this~way.

The practice tends to vary in this way, so it will take many sittings and much effort before you become proficient. Once you are, the mind will let go of the external world and remain undisturbed. Mind-objects from the outside will be unable to penetrate inside the mind and disturb it. Once they are unable to penetrate inside, you will see the mind. You will see the mind as one object of awareness, the breath as another and mind-objects as another. They will all be present within the field of awareness, centred at the tip of your nose. Once mindfulness is firmly established with the in-breath and out-breath, you can continue to practise at your ease. As the mind becomes calm, the breath, which was originally coarse, correspondingly becomes lighter and more refined. The object of mind also becomes increasingly subtle and refined. The body feels lighter and the mind itself feels progressively lighter and unburdened. The mind lets go of external mind-objects and you continue to observe internally.

From here onwards your awareness will be turned away from the world outside and directed inwards to focus on the mind. Once the mind has gathered together and becomes concentrated, maintain awareness at that point where the mind becomes focused. As you breathe, you will see the breath clearly as it enters and leaves, mindfulness will be sharp, and awareness of mind-objects and mental activity will be clearer.

At that point you will see the characteristics of morality, meditation and wisdom, and the way in which they merge together. This is known as the unification of the factors of the Path.\footnote{This refers to the Noble Eightfold Path or the Middle Way, which the Buddha taught as the means to liberation. When the steps towards liberation are emphasized it is referred to as \textit{magga}, the Path. Its eight factors are right view, right intention, right speech, right action, right livelihood, right effort, right mindfulness and right concentration. These eight factors are divided into three groups that are mentioned frequently in this booklet: virtuous conduct or morality, meditation and wisdom.} Once this unification occurs, your mind will be free from all forms of agitation and confusion. It will become one-pointed and this is what is known as \textit{Sam\=adhi}. When you focus attention in just one place, in this case the breath, you gain clarity and awareness because of the uninterrupted presence of mindfulness. As you continue to see the breath clearly, mindfulness will become stronger and the mind will become more sensitive in many different ways. You will see the mind in the centre of that place (the breath), one-pointed, with awareness focused inwards rather than turning towards the world outside. The external world gradually disappears from your awareness, and the mind will no longer be going to perform any work on the outside. It is as if you have come inside your house, where all your sense faculties have come together to form one compact unit. You are at your ease and the mind is free from all external objects. Awareness remains with the breath, and over time it will penetrate deeper and deeper inside, becoming progressively more refined. Ultimately awareness of the breath becomes so refined that the sensation of the breath seems to disappear. You could say either that awareness of the sensation of the breath has disappeared, or that the breath itself has disappeared. Then there arises a new kind of awareness, awareness that the breath has disappeared. In other words, awareness of the breath becomes so refined that it is difficult to define it.

So it might be that you are just sitting there and there is no breath. Really the breath is still there, but it has become so refined that it seems to have disappeared. Why? Because the mind is at its most refined, with a special kind of knowing. All that remains is the knowing. Even though the breath has vanished, the mind is still concentrated with the knowledge that the breath is not there. As you continue, what should you take up as the object of meditation? Take this very knowing as the meditation object -- in other words, the knowledge that there is no breath -- and sustain this. You could say that a specific kind of knowledge has been established in the mind. At this stage you should make the mind unshakeable in its concentration and be especially mindful. Some people become startled when they notice that the breath has disappeared, because they're used to having the breath there. When it appears that the breath has gone, you might panic or become afraid that you are going to die. Here you must establish the understanding that it is just the nature of the practice to progress in this way.

At this point some people may have doubts arising, because it is here that a vision or mental image (\textit{nimitta}) may arise. These can be of many kinds, including both forms and sounds. It is here that all sorts of unexpected things can arise in the course of the practice. If mental images do arise (some people have them, some don't), you must understand them in accordance with the truth. Don't doubt or allow yourself to become alarmed.

What will you observe as the object of meditation now? Observe this feeling that there is no breath and sustain it as the object of awareness as you continue to meditate. The Buddha described this as the firmest, most unshakeable form of \textit{Sam\=adhi}. There is just one firm and unwavering object of mind. When your practice of \textit{Sam\=adhi} reaches this point, many unusual and refined changes and transformations will be taking place within the mind, of which you may be aware. The sensation of the body will feel at its lightest or might even disappear altogether. You might feel as if you are floating in mid-air and seem to be completely weightless. It might be as if you are in the middle of space, and wherever you direct your sense faculties, they don't seem to register anything at all. Even though you know the body is still sitting there, you experience complete emptiness. This feeling of emptiness can be quite strange.

As you continue to practise, understand that there is nothing to worry about. Establish this feeling of being relaxed and unworried, securely in the mind. Once the mind is concentrated and one-pointed, no mind-object will be able to penetrate or disturb it, and you will be able to sit like this for as long as you want. You will be able to sustain concentration without any feelings of pain or discomfort. Having developed \textit{Sam\=adhi} to this level, you will be able to enter or leave it at will. When you do leave it, it's at your ease and convenience. You withdraw at your ease, rather than because you are feeling lazy, unenergetic or tired. You withdraw from \textit{Sam\=adhi} because it is the appropriate time to withdraw, and you come out of it at your will.

This is \textit{Sam\=adhi}: you are relaxed and at ease. You enter and leave it without any problems. The mind and heart are at ease. If you genuinely have \textit{Sam\=adhi} like this, it means that sitting in meditation and entering \textit{Sam\=adhi} for just thirty minutes or an hour will enable you to remain cool and peaceful for many days afterwards. Experiencing the effects of \textit{Sam\=adhi} like this for several days has a purifying effect on the mind -- whatever you experience will become an object for contemplation. This is where the practice really begins. It's the fruit which arises as \textit{Sam\=adhi} matures.

\textit{Sam\=adhi} performs one function, that of calming the mind, while morality and wisdom perform others. These characteristics on which you are focusing attention and which you are developing in the practice are linked, forming a circle. This is the way they manifest in the mind. Morality, \textit{Sam\=adhi} and wisdom arise and mature from the same place. Once the mind is calm, it will become progressively more restrained and composed due to the presence of wisdom and the power of \textit{Sam\=adhi}. As the mind becomes more composed and refined, this gives rise to an energy which acts to purify our conduct. Greater purity of our morality facilitates the development of stronger and more refined \textit{Sam\=adhi}, and this in turn supports the maturing of wisdom. They assist each other in this way. Each aspect of the practice acts as a supporting factor for each of the others -- in the end these terms become synonymous. As these three factors continue to mature together, they form one complete circle, ultimately giving rise to the Path (\textit{magga})\textit{. Magga} is a synthesis of these three functions of the practice, working smoothly and consistently together. As you practise you have to preserve this energy. It is the energy which will give rise to \textit{vipassana} (insight) or wisdom. Having reached this stage, where wisdom is already functioning in the mind, independent of whether the mind is peaceful or not, wisdom will provide a consistent and independent energy in the practice. You see that when the mind is not peaceful, you shouldn't cling to that, and even when it \textit{is} peaceful, you shouldn't cling to that either. Having let go of the burden of such concerns, the heart will accordingly feel much lighter. Whether you experience pleasant mind-objects or unpleasant mind-objects, you will remain at ease. The mind will remain peaceful in this way.

Another important thing to see is that when you stop doing formal meditation practice, if there is no wisdom functioning in the mind, you will give up the practice altogether, without any further contemplation or development of awareness of the work which still has to be done. In fact, when your mind comes out of \textit{Sam\=adhi,} you should know clearly in the mind that you have withdrawn. Having withdrawn, continue to conduct yourself in a normal manner. Maintain mindfulness and awareness at all times. It isn't that you only practise meditation in the sitting posture; \textit{Sam\=adhi} means that the mind is always firm and unwavering. Make the mind firm and steady as you go about your daily life, and maintain this sense of steadiness as the object of awareness at all times. You must practise mindfulness and clear comprehension continuously. After you get up from the formal sitting practice and go about your business -- walking, riding in cars and so on -- whenever your eyes see a form or your ears hear a sound, maintain awareness. As you experience mind-objects which give rise to liking and disliking, try consistently to maintain awareness of the fact that such mental states are impermanent and uncertain. In this way the mind will remain calm and in a state of `normality'.

As long as the mind is calm, use it to contemplate mind-objects. Contemplate the whole of your physical form, the body. You can do this at any time and in any posture, whether doing formal meditation practice, relaxing at home, out at work, or in whatever situation you find yourself. Keep the meditation and the reflection going at all times. Just going for a walk and seeing dead leaves on the ground under a tree can provide an opportunity to contemplate impermanence. Both we and the leaves are the same: when we get old we shrivel up and die. Other people are all the same. This is raising the mind to the level of \textit{vipassana}, contemplating the truth of the way things are the whole time. Whether walking, standing, sitting or lying down, mindfulness is sustained evenly and consistently. This is practising meditation correctly -- you have to be following the mind closely, checking it at all times.

It is now seven o'clock in the evening, and we have been practising meditation together for an hour, establishing the mind in the here and now. When we end our meditation session, it might be that your mind will stop practising completely and will not carry on by contemplating. That's the wrong way to do it. When we stop, all that should stop is the formal aspect of sitting meditation in a group. You should continue practising and developing awareness consistently, without letting up.

I have often taught that not practising consistently is like dripping water. The practice is not a continuous uninterrupted flow. It is like individual drops of water. Mindfulness is not sustained evenly. The important point is that the mind does the practice and nothing else. The body doesn't do it. The mind does the work, the mind does the practice. If you understand this clearly, you will see that you don't necessarily have to be doing formal sitting meditation in order for the mind to know \textit{Sam\=adhi}. The mind is the one who does the practice. You have to experience and understand this for yourself, in your own mind.

Once you do see this for yourself, you will be developing awareness in the mind at all times and in all postures. If you are maintaining mindfulness as an even and unbroken flow, it is as if the drops of water have joined to form a smooth and continuous flow of running water. Mindfulness is present from moment to moment, and accordingly there will be\linebreak\ awareness of mind-objects at all times. If the mind is restrained and composed with uninterrupted mindfulness, you will know each time that wholesome and unwholesome mental states arise. You will know the mind that is calm and the mind that is confused and agitated. Wherever you go you will be practising like this. If you train the mind in this way, it means your meditation will mature quickly and successfully.

Please don't misunderstand. These days it is common for people to go on \textit{vipassana} courses for three or seven days, where they don't have to speak or do anything but meditate. Maybe you have gone on a silent meditation retreat for a week or two, afterwards returning to your normal daily life. You might have left thinking that you've `done \textit{vipassana}', and because you feel that you know what it's all about, carried on going to parties and discos, and indulging in different forms of sensual delight. When you do this, what happens? There won't be any of the fruits of \textit{vipassana} left by the end of it. If you go and do all sorts of unskilful things, which disturb and upset the mind, wasting everything, then the next year go back again and do another retreat for seven days or a few weeks, then come out and carry on with the parties, discos and drinking, that isn't true practice. It isn't Dhamma practice or the path to progress.

You need to make an effort to renounce. You must contemplate until you see the harmful effects which come from such behaviour. See the harm in drinking and going out on the town. Reflect and see the harm inherent in all the different kinds of unskilful behaviour in which you indulge, until it becomes fully apparent. This will provide the impetus for you to take a step back and change your ways. Then you will find some real peace. To experience peace of mind, you have to see clearly the disadvantages and danger in such forms of behaviour. This is practising in the correct way. If you do a silent retreat for seven days, where you don't have to speak to or get involved with anybody, and then go chatting, gossiping and over-indulging for another seven months, how will you gain any real or lasting benefit from those seven days of practice?

I would encourage all the people here, who are practising to develop awareness and wisdom, to understand this point. Try to practise consistently. See the disadvantages of practising insincerely and inconsistently, and try to sustain a more dedicated and continuous effort in the practice. Really take it to this extent. It can then become a realistic possibility that you might put an end to the impurities of the mind. But that style of not speaking and not playing around for seven days, followed by six months of complete sensual indulgence, without any mindfulness or restraint, will just lead to the squandering of any gains made from the meditation -- there won't be anything left. It is as if you were to go to work for a day and earn 200 \textit{Baht,} but then went out and spent 300 baht on food and things in the same day. How would you ever save any money? It would all be gone. It's just the same with meditation.
