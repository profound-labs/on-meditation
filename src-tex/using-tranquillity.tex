
Calming the mind means finding the right balance. If you try to force your mind too much it goes too far; if you don't try enough it misses the point of balance, it doesn't get there.

Normally the mind isn't still, it's moving all the time. We must strengthen the mind. Making the mind strong and making the body strong are not the same. To make the body strong we have to exercise it, to push it in order to strengthen it, but to make the mind strong means to make it peaceful, not to go thinking of this and that. For most of us the mind has never been peaceful, it has never had the energy of \textit{Sam\=adhi}, so we must establish it within certain boundaries. We sit in meditation, staying with the `one who knows'.

If we force our breath to be too long or too short, we're not balanced and the mind won't become peaceful. It's like when we first start to use a pedal sewing machine. At first we just practise pedalling the machine to get our coordination right, before we actually sew anything. Following the breath is similar. We don't get concerned over how long or short, weak or strong it is, we just note it. We simply let it be, following the natural breathing.

When it's balanced, we take up the breathing as our meditation object. When we breathe in, the beginning of the breath is at the tip of the nose, the middle of the breath at the chest and the end of the breath at the abdomen. This is the path of the breath. When we breathe out, the beginning of the breath is at the abdomen, the middle at the chest and the end at the tip of the nose. Simply take note of this path of the breath at the tip of the nose, the chest and the abdomen, then at the abdomen, the chest and the tip of the nose. We take note of these three points in order to make the mind firm, to limit mental activity so that mindfulness and self-awareness can easily arise.

\looseness=1
When our attention settles on these three points, we can let them go and note the in-breathing and out-breathing, concentrating solely at the tip of the nose or the upper lip, where the air passes on its in and out passage. We don't have to follow the breath, just establish mindfulness in front of us at the tip of the nose, and note the breath at this one point -- entering, leaving, entering, leaving. There's no need to think of anything special, just concentrate on this simple task for now, having continuous presence of mind. There's nothing more to do, just breathing in and out. Soon the mind becomes peaceful, the breath refined. The mind and body become light. This is the right state for the work of meditation.

When sitting in meditation, the mind becomes refined, but whatever state it's in we should try to be aware of it, to know it. Mental activity is there, together with tranquillity. There is the action of bringing the mind to the theme of contemplation (\textit{vitakka}). If there is not much mindfulness, there will be not much \textit{vitakka}. Then the contemplation around that theme (\textit{vicara}) follows. Various weak mental impressions may arise from time to time but our self-awareness is the important thing -- whatever may be happening, we know it continuously. As we go deeper we are constantly aware of the state of our meditation, knowing whether or not the mind is firmly established. Thus both concentration and awareness are present.

To have a peaceful mind does not mean that nothing is happening; mental impressions do arise. For instance, when we talk about the first level of the peaceful mind (i.e. \textit{jh\=ana}, absorption), we say it has five factors. Along with \textit{vitakka} and \textit{vicara}, rapture (\textit{piti}) arises with the theme of contemplation, and then happiness (\textit{sukha}). These four things all come together in the mind established in tranquillity. They are as one state.

\looseness=1
The fifth factor is one-pointedness. You may wonder how there can be one-pointedness when there are all these other factors as well. This is because they all become unified on that foundation of tranquillity. Together they are called a state of \textit{Sam\=adhi}. They are not everyday states of mind, they are factors of absorption. There are these five characteristics, but they do not disturb the basic tranquillity. There is \textit{vitakka} (the bringing of the mind to the theme of contemplation), but it does not disturb the mind. \textit{Vicara} (contemplation around that theme) and then rapture and happiness arise, but do not disturb the mind. The mind is therefore as one with these factors. The first level of absorption in peace is like this.

We don't have to call it first \textit{jh\=ana}, second \textit{jh\=ana}, third \textit{jh\=ana} and so on, let's just call it a peaceful mind. As the mind becomes progressively calmer it will dispense with \textit{vitakka} and \textit{vicara}, leaving only rapture and happiness. Why does the mind discard \textit{vitakka} and \textit{vicara}? This is because as the mind becomes more refined, the activities of \textit{vitakka} and \textit{vicara} are too coarse to remain. At this stage, as the mind lets go of \textit{vitakka} and \textit{vicara}, feelings of great rapture can arise, tears may flow. But as the \textit{Sam\=adhi} deepens, rapture too is discarded, leaving only happiness and one-pointedness. Finally even happiness goes and the mind reaches its greatest refinement. There are only equanimity and one-pointedness. All else has been left behind. The mind stands unmoving.

This can happen once the mind is peaceful. You don't have to think a lot about it, it just happens by itself when the causal factors are ripe. This is called the energy of a peaceful mind. In this state the mind is not drowsy; the five hindrances, sense desire, aversion, restlessness, dullness and doubt, have all fled. But if mental energy is still not strong and mindfulness is weak intruding mental impressions, will occasionally arise. The mind is peaceful but it's as if there's a `cloudiness' within the calm. The mind tends to play tricks within these levels of tranquillity. Mental images will sometimes arise when the mind is in this state, through any of the senses, and the meditator may not be able to tell exactly what is happening. `Am I sleeping? No. Is it a dream? No, it's not a dream\ldots{}' It's not a normal sort of drowsiness, though, some impressions will manifest -- maybe we'll hear a sound or see a dog or something. It's not really clear, but it's not a dream either. This is because the five factors of a peaceful mind have become unbalanced and weak.

These impressions arise from a middling sort of tranquillity; but if the mind is truly calm and clear we don't doubt the various mental impressions or imagery which arise. Questions like: `Did I drift off, then? Was I sleeping? Did I get lost?' don't arise, for they are characteristics of a mind which is still doubting. `Am I asleep or awake?' Here the mind is fuzzy. This is the mind getting lost in itself. It's like the moon going behind a cloud. You can still see the moon, but the clouds covering it render it hazy. It's not like the moon which has emerged from behind the clouds clear, sharp and bright.

When the mind is peaceful and established firmly in mindfulness and self-awareness, there will be no doubt concerning the various phenomena which we encounter. The mind will truly be beyond the hindrances. We will clearly know everything which arises in the mind as it is. We will not doubt, because the mind is clear and bright. The mind which reaches \textit{Sam\=adhi} is like this.

Some people find it hard to enter \textit{Sam\=adhi} because they don't have the right tendencies. There is \textit{Sam\=adhi}, but it's not strong or firm. However, one can attain peace through the use of wisdom, through contemplating and seeing the truth of things, solving problems that way. This is using wisdom rather than the power of \textit{Sam\=adhi}. To attain calm in practice, it's not necessary to be sitting in meditation, for instance. Just ask yourself: `What is that?' and solve your problem right there! A person with wisdom is like this. Perhaps he or she can't really attain high levels of \textit{Sam\=adhi}, although there must be some, just enough to cultivate wisdom. It's like the difference between farming rice and farming corn. One can depend on rice more than corn for one's livelihood. Our practice can be like this: we can depend more on wisdom to solve problems. When we see the truth, peace~arises.

The two ways are not the same. Some people have insight and are strong in wisdom, but do not have much \textit{Sam\=adhi}. When they sit in meditation they aren't very peaceful. They tend to think a lot, contemplating this and that, until eventually they contemplate happiness and suffering and see the truth of them. Some incline more towards this than \textit{Sam\=adhi}. Whether standing, walking, sitting or lying down, enlightenment can take place. Through seeing, through relinquishing, they attain peace. They attain peace through knowing the truth, through going beyond doubt, because they have seen it for themselves. Other people have less wisdom, but their \textit{Sam\=adhi} is very strong. They can enter very deep \textit{Sam\=adhi} quickly, but because they don't have much wisdom, they \mbox{cannot} catch their defilements in time, they don't know them. They can't solve their problems.

Regardless of which approach we use, we need to do away with wrong thinking, leaving only right view. We need to get rid of confusion, leaving only peace. Either way we end up at the same place. There are these two sides to practice, but these two things, calm and insight, go together. We can't do away with either of them. They must go together.

That which watches over the various factors which arise in meditation is mindfulness. This mindfulness is a condition which, through practice, can help other factors to arise. Mindfulness is life. When we don't have mindfulness, when we are heedless, it's as if we are dead. If we have no mindfulness, then our speech and actions have no meaning. Mindfulness is simply recollection. It's a cause for the arising of self-awareness and wisdom. Whatever virtues we have cultivated are imperfect if lacking in mindfulness. Mindfulness is that which watches over us while standing, walking, sitting and lying down. Even when we are no longer in \textit{Sam\=adhi}, mindfulness should be present throughout.
