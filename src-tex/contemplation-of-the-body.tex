
In training the mind, it is crucial to overcome sceptical doubt. Doubt and uncertainty are powerful obstacles that must be dealt with. Investigation of the `first three fetters' -- personality view, blind attachment to rules and practices and sceptical doubt\footnote{One of the ways in which the Buddha outlined the path to enlightenment was by means of the gradual abandoning of certain `fetters', the obstructions of the mind that bind us to suffering. The first three of these are \textit{sakk\=aya-di\d{t}\d{t}hi} (personality view); \textit{s\={\i}labbata-par\=am\=asa} (blind attachment to rules and practices) and \textit{vicikicch\=a} (sceptical doubt).} -- is the way out of attachment practised by the Enlightened Ones. But at first you just understand these defilements from the books -- you still lack insight into how things truly are.

Investigating personality view is the way to go beyond the delusion that identifies the body as a self. This includes the attachment to viewing your own and other people's bodies as having a solid self. Personality view refers to this thing you call yourself. It means attachment to the view that the body is a self. You must investigate this view until you gain a new understanding and can see the truth that attachment to the body is defilement, and that it obstructs the minds of all human beings from gaining insight into the Dhamma.

\looseness=1
For this reason, before anything else the preceptor will instruct each new candidate for ordination to investigate five meditation objects: hair of the head, hair of the body, nails, teeth and skin. It is through contemplation and investigation that you develop insight into personality view. These objects form the most immediate basis for the attachment that creates the delusion of personality view. Contemplating them leads to the direct examination of personality view, and provides the means by which each generation of men and women who take up the instructions of the preceptor upon entering the monastic community can actually transcend personality view. But in the beginning you remain deluded, without insight, and hence are unable to penetrate personality view and see the truth of the way things are. You fail to see the truth because you still have a firm and unyielding attachment. It's this attachment that sustains the delusion.

The Buddha taught us to transcend delusion. The way to transcend it is through clearly seeing the body for what it is. You must see with penetrating insight that the true nature of both your own body and other people's bodies is essentially the same. There is no fundamental difference between peoples' bodies. The body is just the body; it's not a being, a self, yours or theirs. A body exists: you label it and give it a name. Then you attach and cling to it, with the view that it is your body or his or her body. You attach to the view that the body is permanent, and that it is something clean and pleasant. This attachment goes deep into the mind. This is the way that the mind clings to the body.

Personality view means that you are still caught in doubt and uncertainty about the body. Your insight hasn't fully penetrated the delusion that sees the body as a self. As long as the delusion remains, you call the body a self or \textit{atta}, and interpret your entire experience from the viewpoint that there is a solid, enduring entity which you call the self. You are so completely attached to the conventional way of viewing the body as a self that there is no apparent way of seeing beyond it. But clear understanding according to the truth of the way things are means you see the body as just that much: the body is just the body. With insight, you see the body as merely that, and this wisdom counteracts the delusion of the sense of self. This insight that sees the body as just that much leads to the destruction of attachment, through the gradual uprooting and letting go of delusion.

Practise contemplating the body as being just that much, until it is quite natural to think to yourself: `Oh, the body is merely the body. It's just that much.' Once this way of reflection is established, as soon as you say to yourself that it's just that much, the mind lets go. There is the letting go of attachment to the body. There is the insight that sees the body as merely the body. By sustaining this sense of detachment through continuously seeing the body as merely the body, all doubt and uncertainty are gradually uprooted. As you investigate the body, the more clearly you see it as just the body rather than a person, a being, a me or a them, the more powerful the effect on the mind, culminating in the simultaneous removal of doubt and uncertainty.

Blind attachment to rules and practices, which manifests in the mind as blindly fumbling and feeling around through lack of clarity as to the real purpose of practice, is abandoned simultaneously because it arises in conjunction with personality view. You could say that the three fetters of doubt, blind attachment to rites and practices and personality view are inseparable, and even effectively synonyms for each other. Once you have seen this relationship clearly, when one of the three fetters arises, such as doubt, and you are able to let it go through the cultivation of insight, the other two fetters are automatically abandoned at the same time. They are extinguished together. Simultaneously you let go of personality view and the blind attachment that is the cause of fumbling and fuzziness of intention over different practices. You see each of them as one part of your overall attachment to the sense of self, which is to be abandoned. You must repeatedly investigate the body and break it down into its component parts. As you see each part as it truly is, the perception of the body as a solid entity or self is gradually eroded. You have to put continuous effort into this investigation of the truth, without letting up.

A further aspect of mental development that leads to clearer and deeper insight is meditating on an object to calm the mind down. The calm mind is the mind that is firm and stable in \textit{Sam\=adhi}. This can be \textit{khanika-Sam\=adhi} (momentary concentration), \textit{upacara}-\textit{Sam\=adhi} (`neighbourhood' or intermediate concentration) or \textit{appana-Sam\=adhi} (full absorption). The level of concentration is determined by the refinement of consciousness from moment to moment, as you train the mind to maintain awareness on a meditation object.

In momentary concentration, the mind unifies for just a short space of time. It calms down in \textit{Sam\=adhi}, but after momentarily gathering together, it immediately withdraws from that peaceful state. As concentration becomes more refined in the course of meditation, many similar characteristics of the tranquil mind are experienced at each level, so each one is described as a level of \textit{Sam\=adhi}, whether it is momentary or neighbourhood concentration, or absorption. At each level the mind is calm, but the depth of the \textit{Sam\=adhi} varies and the nature of the peaceful mental state experienced differs. On one level the mind is still subject to movement and can wander, but it moves around within the confines of the concentrated state. It doesn't get caught into activity that leads to agitation and distraction. Your awareness might follow a wholesome mental object for a while, before returning to settle down at a point of stillness where it remains for a period.

You could compare the experience of momentary concentration with a physical activity like taking a walk somewhere: you might walk for a period before stopping for a rest, and after having rested start walking again until it's time to stop for another rest. Although you interrupt the journey periodically to stop walking and take rests, remaining completely still each time, this is just a temporary stillness of the body. After a short space of time you have to start moving again to continue the journey. This is what happens within the mind as it experiences such a level of concentration.

The practice of meditation by focusing on an object to calm the mind and reach a level of calm where the mind is firm in \textit{Sam\=adhi}, but with some mental movement still occurring, is known as neighbourhood concentration. In neighbourhood concentration the mind can still move around. This movement takes place within certain limits; the mind doesn't move beyond them. The boundaries within which the mind can move are determined by the firmness and stability of the concentration. The experience is as if you alternate between a state of calm and a certain amount of mental activity. The mind is calm some of the time and active for the rest. Within that activity a certain level of calm and concentration still persists, but the mind is not completely still or immovable. It is still thinking a little and wandering about. It's as if you are wandering around inside your own home. You wander around within the limits of your concentration, without losing awareness and moving outdoors, away from the meditation object. The movement of the mind stays within the bounds of wholesome mental states. It doesn't get caught into any mental proliferation based on unwholesome mental states. Any thinking remains wholesome. Once the mind is calm, it necessarily experiences wholesome mental states from moment to moment. During the time when it is concentrated, the mind only experiences wholesome mental states and periodically settles down to become completely still and one-pointed on its object. So the mind still experiences some movement, circling around its object. It can still wander. It might wander around within the confines set by the level of concentration, but no real harm arises from this movement because the mind is calm in \textit{Sam\=adhi}. This is how the development of the mind proceeds in the course of practice.

When the mind enters absorption it calms down and is stilled to a level where it is at its most subtle and skilful. Even if you experience sense impingement from the outside, such as sounds and physical sensations, it remains external and is unable to disturb the mind. You might hear a sound, but it won't disturb your concentration. There is the hearing of the sound, but the experience is as if you don't hear anything. There is awareness of the impingement, but it's as if you are not aware. This is because you let go. The mind lets go automatically. Concentration is so deep and firm that you let go of attachment to sense impingement quite naturally. The mind can absorb into this state for long periods. Having stayed inside for an appropriate amount of time, it then withdraws.

Sometimes a mental image (\textit{nimitta}) of some aspect of your own body can appear as you withdraw from such a deep level of concentration. It might be a mental image displaying an aspect of the unattractive nature of your body that arises into consciousness. As the mind withdraws from the refined state, the image of the body appears to emerge and expand from within the mind. Any aspect of the body could come up as a mental image and fill the mind's eye at that point. Images that come up in this way are extremely clear and unmistakable. You must have genuinely experienced very deep tranquillity for them to arise. You see them absolutely clearly, even though your eyes are closed. If you open your eyes you can't see them, but with eyes shut and the mind absorbed in \textit{Sam\=adhi}, you can see such images as clearly as if viewing the object with eyes wide open. You can even experience a whole train of consciousness, where from moment to moment the mind's awareness is fixed on images expressing the unattractive nature of the body. The appearance of such images in a calm mind can become the basis for insight into the impermanent nature of the body, as well as its unattractive, unclean and unpleasant nature or the complete lack of any real self or essence within it.

\looseness=1
When these kinds of special knowledge arise they provide the basis for skilful investigation and the development of insight. You bring this kind of insight deep into your heart. As you do this more and more, it becomes the cause for insight knowledge to arise by itself. Sometimes, when you turn your attention to reflecting on the unattractiveness and loathsomeness of the body, images of different unattractive aspects of the body can manifest in the mind automatically. These images are clearer than any you could try to summon up with your imagination, and they lead to insight of a far more penetrating nature than that gained through the ordinary kind of discursive thinking. This kind of clear insight has such a striking impact that the activity of the mind is brought to a stop, followed by the experience of a deep sense of dispassion. It is so clear and piercing because it originates from a completely peaceful mind. Investigating from within a state of calm leads you to clearer and clearer insight, the mind becoming more peaceful as it is increasingly absorbed in contemplation. The clearer and more conclusive the insight, the deeper the mind penetrates with its investigation, constantly supported by the calm of \textit{Sam\=adhi}. This is what the work of meditation practice involves. Continuous investigation in this way helps you to let go repeatedly of attachment to personality view and ultimately destroy it. It puts an end to all remaining doubt and uncertainty about this heap of flesh we call the body, and brings about the letting go of blind attachment to rules and practices.

Even in the event of serious illness, tropical fevers or different health problems that normally have a strong physical impact and shake the body up, your \textit{Sam\=adhi} and insight will remain firm and imperturbable. Your understanding and insight allow you to make a clear distinction between mind and body -- the mind is one phenomenon, the body another. When you see body and mind as completely and indisputably separate from each other, this means that the practice of insight has brought you to the point where your mind sees for certain the true nature of the body.

Seeing the way the body truly is clearly and beyond doubt from within the calm of \textit{Sam\=adhi} leads the mind to experience a strong sense of world-weariness and turning away. This turning away comes from the sense of disenchantment and dispassion that arises as the natural result of seeing the way things are. It's not a turning away that comes from ordinary worldly moods such as fear, revulsion or other unwholesome qualities like envy or aversion. It's not coming from the same root of attachment as those defiled mental states. This is a turning away that has a spiritual quality to it, and has a different effect on the mind from the normal moods of boredom and weariness experienced by ordinary unenlightened human beings.

\looseness=1
When ordinary people are weary and fed up, they usually become caught up into moods of aversion, rejection and seeking to avoid. The experience of insight is not the same. The sense of world-weariness that grows with insight, leads to a detachment, turning away and aloofness that come naturally from investigating and seeing the truth of the way things are. It is free from attachment to a sense of a self which attempts to control and force things to go according to its desires. Instead you let go with acceptance of the way things are. The clarity of insight is so strong that you no longer experience any sense of a self that has to struggle against the flow of its desires or endure through attachment. The three fetters of personality view, doubt and blind attachment to rules and practices that are normally present, underlying the way you view the world, can't now delude you or cause you to make any serious mistakes in~practice.

This is the very beginning of the path, the first clear insight into ultimate truth, and it paves the way for further insight. You could describe it as penetrating the Four Noble Truths. The Four Noble Truths are to be realized through insight. Every monk and nun who has ever realized them has experienced such insight into the truth of the way things are. You know suffering, know the cause of suffering, know the cessation of suffering and know the path leading to the cessation of suffering. The understanding of each Noble Truth emerges at the same place within the mind. They come together and harmonize as the factors of the Noble Eightfold Path, which the Buddha taught are to be realized within the mind. As the path factors converge in the centre of the mind, they cut through any doubts and uncertainty you may still have concerning the way of practice.

The investigation and development of insight into the Dhamma give rise to this profound peace of mind. Once it is gained this clear and penetrating insight is sustained at all times, whether you are sitting in meditation with your eyes closed or doing something with your eyes open. Whatever situation you find yourself in, be it formal meditation or not, the clarity of insight remains. When you have unwavering mindfulness of the mind, you don't forget yourself. Whether you are standing, walking, sitting or lying down, the awareness within makes it impossible to lose mindfulness. It's a state of awareness that prevents you from forgetting yourself. Mindfulness has become so strong that it is self-sustaining, to the point where it becomes natural for the mind to be that way. These are the results of training and cultivating the mind, and here is where you go beyond doubt. You have no doubts about the future, you have no doubts about the past, so you have no need to doubt about the present either, although you still have awareness that there are such things as past, present and future. You are aware of the existence of time and there is the reality of the past, present and future, but you are no longer concerned or worried about them.

Why are you no longer concerned? All those things that took place in the past have already happened. The past has already gone. All that is arising in the present is the result of causes that lay in the past. An obvious example of this is to say that if you don't feel hungry now, it's because you have already eaten at some time in the past. The lack of hunger in the present is the result of actions performed in the past. If you know your experience in the present, you can know the past. Eating a meal was the cause from the past that resulted in your feeling at ease or energetic in the present, and this provides the cause of your being active and working in the future. So the present is providing causes that will bring results in the future. Thus the past, present and future can be seen as one and the same. The Buddha called it \textit{eko dhammo} --  the unity of the Dhamma. It isn't many different things; there is just this much. When you see the present, you see the future. By understanding the present you understand the past. Past, present and future make up a chain of continuous cause and effect, and hence are constantly flowing on from one to the other. There are causes from the past that produce results in the present, and those results are already producing causes for the future. This process of cause and result applies to practice in the same way. You experience the fruits of having trained the mind in \textit{Sam\=adhi} and insight, and they necessarily make the mind wiser and more skilful.

The mind completely transcends doubt. You are not uncertain or speculating about anything any longer. The lack of doubt means you no longer fumble around or have to feel your way through the practice. As a result you live and act in accordance with nature. You live in the world in the most natural way. That means living in the world peacefully: you are able to find peace even in the midst of that which is not peaceful. It means you are fully able to live in the world. You are able to live in the world without creating any problems. The Buddha lived in the world and was able to find true peace of mind within the world. As practitioners of the Dhamma, you must learn to do the same. Don't get lost in and attached to perceptions about things being this way or that way. Don't attach or give undue importance to any perceptions that are still deluded. Whenever the mind becomes stirred up, investigate and contemplate the cause. When you aren't making any suffering for yourself out of things, you are at ease. When there are no issues causing mental agitation, you remain equanimous, that is, you continue to practise normally, with mental equanimity maintained by the presence of mindfulness and an all-round awareness. You maintain a sense of self-control and equilibrium. If any matter arises and prevails upon the mind, you immediately take hold of it for thorough investigation and contemplation. If there is clear insight at that moment, you penetrate the matter with wisdom and prevent it creating any suffering in the mind. If there is not yet clear insight, you let the matter go temporarily through the practice of tranquillity meditation, and don't allow the mind to attach. At some point in the future your insight will certainly be strong enough to penetrate it, because sooner or later you will develop insight powerful enough to comprehend everything that still causes attachment and suffering.
