
When sitting in meditation and making the mind peaceful, you don't have to think about too much. Just focus on the mind and nothing else. Don't let the mind shoot off to the left or to the right, to the front or behind, above or below. Our only duty is to practice mindfulness of the breathing. But first, fix your attention at the head and move it down through the body to the tips of the feet, and then back up to the crown of the head. Pass your awareness down through the body, observing with wisdom. We do this to gain an initial understanding of the way the body is right now. Then begin the meditation, noting that at this time your sole duty is to observe the inhalations and exhalations. Don't force the breath to be any longer or shorter than normal, just allow it to continue easily. Don't put any pressure on the breath, rather let it flow evenly, letting go with each in-breath and out-breath. You must understand that you are letting go as you do this, but there should still be awareness. You must maintain this awareness, allowing the breath to enter and leave comfortably. There is no need to force the breath, just allow it to flow easily and naturally. Maintain the resolve that at this time you have no other duties or responsibilities. Thoughts about what will happen, what you will know or see during the meditation, may arise from time to time, but once they arise just let them cease by themselves, don't be unduly concerned with them.

During the meditation there is no need to pay attention to anything that arises in the mind. Whenever the mind is affected by any thoughts or moods, whenever there is a feeling or sensation in the mind, just let them go. Whether those thoughts are good or bad is unimportant. It is not necessary to make anything out of them, just let them pass away and return your attention to the breath. Maintain the awareness of the breath entering and leaving in a relaxed way. Don't worry about the breath being either too long or too short. Simply observe it without trying to control or suppress it in any way. In other words, don't attach to anything. Allow the breath to continue as it is, and the mind will become calm. As you continue the mind will gradually lay things down and come to rest, the breath becoming lighter and lighter until it becomes so faint that it seems as if it's not there at all. Both the body and the mind will feel light and energized. All that will remain will be a one-pointed knowing. You could say that the mind has changed and reached a state of calm.

If the mind is agitated, re-establish mindfulness and inhale deeply until there is no space left to store any air, then release it all completely until none remains. Follow this with another deep inhalation until your lungs are full, then release the air again. Do this two or three times, then re-establish concentration. The mind should be calmer. If any more sense impressions cause agitation in the mind, repeat the process on every occasion. Do the same when doing walking meditation: if the mind becomes agitated while walking, stand still, calm the mind, re-establish the awareness of the meditation object and then continue walking. Sitting and walking meditation are in essence the same, differing only in terms of the physical posture.

Sometimes doubts may arise, so you must have mindfulness, be `\textit{the one who knows}', continually following and examining the agitated mind in whatever form it takes. This is what it means to have mindfulness. Mindfulness watches over and takes care of the mind. You must maintain this knowing and not be careless or wander astray, no matter what state the mind takes on.

The trick is to have awareness overseeing the mind. Once the mind is unified with mindfulness, a new kind of awareness will emerge. The mind that has developed calm is held in check by that calm, just like a chicken held in a coop. The chicken is unable to wander outside, but it can still move around within the coop. It doesn't matter that it is walking to and fro, because it stays in the coop. Likewise, the awareness that takes place when the mind has mindfulness and is calm does not cause trouble. None of the thoughts or sensations that take place within the calm mind cause harm or disturbance.

\looseness=1
Some people don't want to experience any thoughts or feelings at all, but this is not right. Feelings arise within the state of calm. The mind is both experiencing feelings and calm at the same time, without being disturbed. When there is calm like this, there are no harmful consequences. Problems occur when the chicken gets out of the coop. For instance, you may be watching the breath entering and leaving and then forget yourself, allowing the mind to wander away from the breath, back home, off to the shops or to any number of different places. Maybe even half an hour may pass before you suddenly realize you're supposed to be practising meditation, and you think: `Oh, what am I doing?' This is when you have to be really careful, because this is when the chicken gets out of the coop -- the mind leaves its base~of~calm.

You must take care to maintain the awareness with mindfulness and try to pull the mind back. Although I use the words `pull the mind back', in fact the mind doesn't really go anywhere. Only the object of awareness has changed. You must make the mind stay right here and now. As long as there is mindfulness there will be presence of mind. It seems as if you are pulling the mind back, but really it hasn't gone anywhere, it has simply changed a little. It seems that the mind goes here and there, but in fact the change occurs at exactly the same spot. Then, when mindfulness is re-established, it is back in a flash. It doesn't come from somewhere else. Understand that it is right here.

When there is total knowing, a continuous and unbroken awareness at each and every moment, this is called presence of mind. If your attention drifts from the breath to other places, the knowing is broken. Whenever there is awareness of the breath, the mind is there. With just the breath and this even and continuous awareness you have presence of mind.

There must be both mindfulness (\textit{sati}) and clear comprehension (\textit{sampajañña}). Mindfulness is recollection; and clear comprehension is self-awareness. Right now you are clearly aware of the breath. This exercise of watching the breath helps mindfulness and clear comprehension to develop together. They share the work. Having both mindfulness and clear comprehension is like having two workers to lift a heavy plank of wood. Suppose there are two people trying to lift some big planks, but the weight is so heavy, they have to strain so hard, that it's almost unbearable. Then another person, imbued with goodwill, sees them and rushes in to help. In the same way, when there is mindfulness and clear comprehension, wisdom will arise at the same place to help out. Then all three of them support each other.

With wisdom there will be an understanding of sense objects. For instance, during the meditation sense objects are experienced and give rise to feelings and moods. You may start to think of a friend, but then wisdom should immediately counter with: `It doesn't matter', `Stop'', or `Forget it'. Or if there are thoughts about where you will go tomorrow, then the response would be: `I'm not interested, I don't want to concern myself with such things'. Maybe you start thinking about other people: then you should think: `No, I don't want to get involved', `Just let go' or: `It's all uncertain and never a sure thing'. This is how you should deal with things in meditation. Recognize them as `not sure, not sure', and maintain this kind of awareness.

You must give up all thinking, the inner dialogue and the doubting. Don't get caught up in these things during the meditation. In the end all that will remain in the mind -- in its purest form -- are mindfulness, clear comprehension and wisdom. Whenever these things weaken, doubts will arise. Try to abandon those doubts immediately, leaving only mindfulness, clear comprehension and wisdom. Try to develop mindfulness like this until it can be maintained at all times. Then you will understand mindfulness, clear comprehension and meditation thoroughly.

If you focus the attention at this point, you will see mindfulness, clear comprehension, the concentrated mind and wisdom together. If you are attracted to or repelled by external sense objects, you will be able to tell yourself: `It's not sure'. Either way they are just hindrances to be swept away until the mind is clean. All that should remain are mindfulness and recollection, clear comprehension and awareness, concentration, the firm and unwavering mind -- and all-round wisdom.
